\chapter{Conclusiones}
El culmen de este Trabajo de Fin de Grado ha supuesto la integración de los conocimientos, así como competencias adquiridas a lo largo de la trayectoria académica. Durante el desarrollo del proyecto he alcanzado una compresión profunda de las tendencias fundamentales entre los parámetros clave del sistema. El análisis iterativo y de la influencia de variables me ha permitido identificar relaciones críticas. Esta capacidad de interpretar y anticipar el comportamiento de los sistemas ante cambios en los parámetros de diseño constituye una de las competencias más valiosas adquiridas durante el trabajo.

%La culminación de este Trabajo de Fin de Grado ha implicado la integración de los conocimientos y competencias adquiridas a lo largo de la trayectoria académica. Para la elaboración del presente documento, se ha atendido a las directrices lingüísticas de la Real Academia Española (RAE) y a los estándares de formato y citación IEEE[Previous Query], con el fin de asegurar la precisión terminológica y la claridad expositiva.

El carácter multidisciplinar del proyecto se ha reflejado en la necesidad de integrar conocimientos de óptica, mecánica orbital, desarrollo de software, etc. La validación de los cálculos mediante campañas de simulación a través del tutor y empleando software profesional, ha garantizado la robustez de los resultados obtenidos y la fiabilidad de las soluciones propuestas.

Tras meses de trabajo se ha conseguido un solución optima que cumple con los requisitos especificados. Uno de los logros más significativos ha sido el desarrollo de un código propio, modular, capaz de facilitar la iteración ágil sobre los parámetros de diseño. Esta herramienta ha permitido explorar de manera eficiente el espacio de soluciones, manteniendo la agilidad necesaria para cambiar parámetros e ir probando diferentes configuraciones, optimizando así el proceso de toma de decisiones.

El trabajo realizado pone de manifiesto la necesidad de adoptar metodologías iterativas en el diseño de sistemas complejos. La experiencia adquirida en la gestión de incertidumbres y en la interpretación crítica de los resultados constituye una base sólida para afrontar retos profesionales en el ámbito de la ingeniería aeroespacial.
