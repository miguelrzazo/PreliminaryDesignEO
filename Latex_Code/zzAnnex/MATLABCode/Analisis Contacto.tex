\begin{minted}[fontsize=\tiny, frame=lines, framesep=1em]{matlab}

%% 1. DEFINICIÓN DE PARÁMETROS DE SIMULACIÓN Y MISIÓN
startTime = datetime('28-Apr-2024 00:00:00', 'TimeZone', 'UTC');
stopTime = startTime + days(7);
sampleTime = 60; % [s]

gsName = 'Fairbanks'; gsLat = 64.84; gsLon = -147.712; minElevation = 10;

% --- Parámetros de la Constelación y Satélite (SSO) ---
N = 2; altitude = 520; earthRadius = 6371;
semiMajorAxis = (earthRadius + altitude) * 1000;
eccentricity = 0.001; inclination = 97.48;
argOfPeriapsis = 0;
mu = 3.986004418e14;
orbitalPeriod_s = 2 * pi * sqrt(semiMajorAxis^3 / mu);
LTAN_target = 6; % [horas] LTAN de las 6:00 AM

% --- Parámetros de Datos y Memoria (en Gigabytes) ---
GSD = 80; areaToMap_km2 = 8080464; numBands = 3; bitsPerBand = 12;
bitsPerPixel = numBands * bitsPerBand;
areaToMap_m2 = areaToMap_km2 * 1e6;
pixelArea_m2 = GSD^2;
numPixels = areaToMap_m2 / pixelArea_m2;
totalDataBits = numPixels * bitsPerPixel;
totalDataToMap_GB = totalDataBits / (8 * 1e9); 
downloadRate_Mbps = 76;
memoryPerSatellite_GB = 1; % Capacidad en GB

%% 2. CREACIÓN DEL ESCENARIO DE SIMULACIÓN
sc = satelliteScenario(startTime, stopTime, sampleTime);

%% 3. DEFINICIÓN DE LA ESTACIÓN DE TIERRA Y SATÉLITES
gs = groundStation(sc, 'Name', gsName, 'Latitude', gsLat, 'Longitude', gsLon, 'MinElevationAngle', minElevation);

fprintf('Calculando RAAN para un LTAN objetivo de %d:00...\n', LTAN_target);
pos_sun_eci = planetEphemeris(juliandate(startTime),'Earth','Sun');
[alpha_sun_rad, ~, ~] = cart2sph(pos_sun_eci(1), pos_sun_eci(2), pos_sun_eci(3));
alpha_sun_deg = rad2deg(alpha_sun_rad);
raan_offset = (LTAN_target - 12) * 15;
raan_calculated = alpha_sun_deg + raan_offset;
fprintf('RAAN calculado: %.2f grados.\n', raan_calculated);

sats = [];
trueAnomalySeparation = 360 / N;
colors = [[0.8500, 0.3250, 0.0980]; [0.0, 0.4470, 0.7410]; [0.4660, 0.6740, 0.1880]; [0.4940, 0.1840, 0.5560]];
for i = 1:N
    trueAnomaly = (i-1) * trueAnomalySeparation;
    satName = sprintf('Sat %d', i);
    sats = [sats, satellite(sc, semiMajorAxis, eccentricity, inclination, raan_calculated, ...
        argOfPeriapsis, trueAnomaly, ...
        "Name", satName, ...
        "OrbitPropagator", "two-body-keplerian")];
end

%% 4. CÁLCULO DE INTERVALOS DE ACCESO
fprintf('Pre-calculando intervalos de acceso para optimizar la simulación...\n');
accessIntervalsAllSats = cell(N, 1);
for i = 1:N
    accessObject = access(sats(i), gs);
    accessIntervalsAllSats{i} = accessIntervals(accessObject);
end
fprintf('Cálculo de acceso completado.\n');


%% 5. ANÁLISIS DE VIABILIDAD BASADO EN TIEMPO DE DESCARGA
fprintf('\n--- Análisis de Viabilidad basado en Tiempo de Descarga ---\n');

% --- Cálculo del tiempo de descarga necesario ---
% Se convierte el total de datos a Megabits (Mb)
totalDataToMap_Mb = totalDataToMap_GB * 8 * 1000;
% Se calcula el tiempo necesario en segundos y horas para descargar todos los datos
tiempoDescargaNecesario_s = totalDataToMap_Mb / downloadRate_Mbps;
tiempoDescargaNecesario_h = tiempoDescargaNecesario_s / 3600;

fprintf('Datos totales a descargar: %.2f GB\n', totalDataToMap_GB);
fprintf('Tasa de descarga: %d Mbps\n', downloadRate_Mbps);
fprintf('Tiempo de descarga NECESARIO: %.2f horas (%.0f segundos).\n', tiempoDescargaNecesario_h, tiempoDescargaNecesario_s);

% --- Cálculo del tiempo de descarga total obtenido por los satélites ---
tiempoDescargaObtenido_s = 0;
for i = 1:N
    if ~isempty(accessIntervalsAllSats{i}) && height(accessIntervalsAllSats{i}) > 0
        % La duración de cada acceso es EndTime - StartTime
        durations = accessIntervalsAllSats{i}.EndTime - accessIntervalsAllSats{i}.StartTime;
        % Se suma la duración total (en segundos) de los accesos para toda la constelación
        tiempoDescargaObtenido_s = tiempoDescargaObtenido_s + sum(seconds(durations));
    end
end
tiempoDescargaObtenido_h = tiempoDescargaObtenido_s / 3600;
fprintf('Tiempo de descarga OBTENIDO (total constelación): %.2f horas (%.0f segundos).\n', tiempoDescargaObtenido_h, tiempoDescargaObtenido_s);

% --- Comprobación de viabilidad basada en el tiempo ---
if tiempoDescargaObtenido_s >= tiempoDescargaNecesario_s
    fprintf('RESULTADO (por tiempo): MISIÓN VIABLE. El tiempo de acceso total es suficiente.\n\n');
else
    deficit_s = tiempoDescargaNecesario_s - tiempoDescargaObtenido_s;
    deficit_h = deficit_s / 3600;
    fprintf('RESULTADO (por tiempo): MISIÓN NO VIABLE. Se necesita más tiempo de acceso.\n');
    fprintf('Déficit de tiempo: %.2f horas (%.0f segundos).\n\n', deficit_h, deficit_s);
end



%% 6. SIMULACIÓN DE MEMORIA
fprintf('Iniciando simulación de memoria en GB...\n');
timeVector = startTime:seconds(sampleTime):stopTime;
memoryState = zeros(N, numel(timeVector));
totalSimSeconds = seconds(stopTime - startTime);
dataGeneratedPerSample_GB = (totalDataToMap_GB / totalSimSeconds) * sampleTime;
downloadPerSample_GB = (downloadRate_Mbps * 1e6 * sampleTime) / (8 * 1e9);

for t = 2:numel(timeVector)
    currentTime = timeVector(t);
    for i = 1:N
        memoryState(i, t) = memoryState(i, t-1);
        memoryState(i, t) = min(memoryState(i, t) + dataGeneratedPerSample_GB, memoryPerSatellite_GB);
        isAccess = any(currentTime >= accessIntervalsAllSats{i}.StartTime & currentTime <= accessIntervalsAllSats{i}.EndTime);
        if isAccess
            memoryState(i, t) = max(0, memoryState(i, t) - downloadPerSample_GB);
        end
    end
end

%% 7. GENERACIÓN DE GRÁFICOS Y COMPROBACIÓN FINAL
elapsedDays = days(timeVector - startTime);
figure('Name', 'Estado de Memoria de Satélites');
hold on;
yline(memoryPerSatellite_GB, '--k', 'Limite de Memoria', 'LineWidth', 1.5);

for i = 1:N
    plot(elapsedDays, memoryState(i,:), 'Color', colors(i,:), 'LineWidth', 2, 'DisplayName', sprintf('Satelite %d', i));
end
hold off;
grid on;
xlabel('Tiempo (días)', 'Interpreter', 'latex');
ylabel('Memoria Ocupada (GB)', 'Interpreter', 'latex');
title(sprintf('Estado de Memoria (Capacidad Maxima: %.0f GB)', memoryPerSatellite_GB), 'Interpreter', 'latex');
legend('show', 'Interpreter', 'latex', 'Location', 'best');
set(gca, 'TickLabelInterpreter', 'latex');
exportgraphics(gca, 'Estado_Memoria_Satelites.png', 'Resolution', 300);

% Comprobación de viabilidad basada en memoria (del código original)
fprintf('\n--- Análisis de Viabilidad basado en Llenado de Memoria ---\n');
if all(max(memoryState,[],2) < memoryPerSatellite_GB * 0.999)
    fprintf('RESULTADO (por memoria): VIABLE. La memoria de los satélites nunca se llena.\n');
else
    fprintf('RESULTADO (por memoria): NO VIABLE. Al menos un satélite ha llenado su memoria.\n');
end


%% 7. GENERACIÓN DE VISUALIZACIONES 3D DEL ESCENARIO
fprintf('\n--- Generando y exportando visualizaciones 3D del escenario ---\n');
generate3DPlot(sc, sats, gs, orbitalPeriod_s, 'Escenario_3D_1_Orbita', colors);
%generate3DPlot(sc, sats, gs, 86400, 'Escenario_3D_1_Dia', colors);
%generate3DPlot(sc, sats, gs, 7*86400, 'Escenario_3D_1_Semana', colors);
%fprintf('Se han generado 3 archivos PNG con las visualizaciones 3D.\n');

%% 8. GENERACIÓN DE TRAZAS 2D CON WORLDMAP
fprintf('\n--- Generando y exportando visualizaciones 2D (worldmap) ---\n');
load coastlines;
generateWorldmapPlot(sats, gs, startTime, orbitalPeriod_s, 'Traza 2D 1 Orbita', colors, coastlat, coastlon, sampleTime);
generateWorldmapPlot(sats, gs, startTime, 86400, 'Traza 2D 1 Dia', colors, coastlat, coastlon, sampleTime);
generateWorldmapPlot(sats, gs, startTime, 7*86400, 'Traza 2D 1 Semana', colors, coastlat, coastlon, sampleTime);
fprintf('Se han generado 3 archivos PNG con las trazas en 2D.\n');

%% --- FUNCIONES AUXILIARES ---
function generateWorldmapPlot(sats, gs, startTime, duration_s, title_str, colors, coastlat, coastlon, sampleTime)
    fig = figure('Name', title_str, 'NumberTitle', 'off', 'Visible', 'off');
    
    worldmap('north america'); 
    geoshow(coastlat, coastlon, 'Color', 'black');
    hold on;
    
    timeVecPlot = startTime:seconds(sampleTime):(startTime + seconds(duration_s));
    
    plotHandles = gobjects(1, numel(sats));
    legendNames = cell(1, numel(sats));

    for i = 1:numel(sats)
        fprintf('Calculando traza 2D para %s (%s)...\n', sats(i).Name, strrep(title_str, '_', ' '));
        lat = zeros(1, numel(timeVecPlot));
        lon = zeros(1, numel(timeVecPlot));
        for t_idx = 1:numel(timeVecPlot)
            pos_geo_point = states(sats(i), timeVecPlot(t_idx), 'CoordinateFrame', 'geographic');
            lat(t_idx) = pos_geo_point(1);
            lon(t_idx) = pos_geo_point(2);
        end
        plotHandles(i) = geoshow(lat, lon, 'DisplayType', 'line', 'Color', colors(mod(i-1, size(colors,1))+1,:), 'LineWidth', 1.5);
        legendNames{i} = strrep(sats(i).Name, '_', '\_');
    end
    
    geoshow(gs.Latitude, gs.Longitude, 'DisplayType', 'point', 'Marker', 'o', 'MarkerEdgeColor', 'r', 'MarkerFaceColor', 'r', 'MarkerSize', 8);
    
    % 1. Título con intérprete LaTeX. Se escapan los guiones bajos.
    title(strrep(title_str, '_', '\_'), 'Interpreter', 'latex');
    
    gridm on; mlabel on; plabel on;
    
    % 2. Se obtiene el manejador de los ejes actuales para modificar las etiquetas.
    ax = gca;
    
    % 3. Se establece el intérprete LaTeX para las etiquetas de los meridianos y paralelos.
    ax.TickLabelInterpreter = 'latex';
    
    legend(plotHandles, legendNames, 'Interpreter', 'latex', 'Location', 'best');
    hold off;
    
    filename = [title_str, '.png'];
    exportgraphics(fig, filename, 'Resolution', 300);
    close(fig);
end



function generate3DPlot(sc, sats, gs, duration_s, title_str, colors)
    fig = uifigure('Name', strrep(title_str, '_', ' '), 'NumberTitle', 'off', 'Visible', 'off');
    v = satelliteScenarioViewer(sc, 'Parent', fig);
    for i = 1:numel(sats)
        groundTrack(sats(i), 'LeadTime', duration_s, 'TrailTime', 0, ...
            'LeadLineColor', colors(mod(i-1, size(colors,1))+1,:), 'LineWidth', 1.5);
    end
    %v.Globe.CameraTarget = [gs.Latitude, gs.Longitude, 0];
    %v.Globe.CameraPosition = [gs.Latitude, gs.Longitude, 20000e3];
    %v.Globe.CameraUpVector = [0 1 0];
    gs.ShowLabel = true;
    %gs.LabelFontSize = 14;
    %drawnow;
    %filename = [title_str, '.png'];
    %exportgraphics(fig, filename, 'Resolution', 300);
    %close(fig);
end

\end{minted}