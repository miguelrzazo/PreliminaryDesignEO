\documentclass[10pt]{beamer}
% ========================
% OPCIONES DE PRESENTADOR
% ========================
%\setbeameroption{show notes on second screen=right} % Mostrar notas en segunda pantalla

% ========================
% PAQUETES
% ========================
\usepackage[spanish]{babel}
\usepackage[utf8]{inputenc}
\usepackage[T1]{fontenc}
\usepackage[backend=biber,style=authoryear]{biblatex}
\addbibresource{referencias.bib} % Usa el mismo archivo que tu trabajo
% Fuentes recomendadas: Alegreya Sans e Inconsolata
\usepackage[sfdefault]{AlegreyaSans} %
\usepackage[scaled=0.85]{inconsolata}
\usepackage{graphicx}
\usepackage{xcolor}
\usepackage{tikz}
\usetikzlibrary{positioning, arrows.meta, calc}
% ========================
% COLORES PERSONALIZADOS
% ========================
\definecolor{mybackground}{RGB}{252,248,244} % Color crema
\definecolor{mytitle}{RGB}{40,40,40} % Gris oscuro para título
\definecolor{myaccent}{RGB}{217,85,59} % Naranja acento
\definecolor{mycode}{RGB}{175,30,45} % Rojo apagado para código o enlaces

\setbeamercolor{background canvas}{bg=mybackground}
\setbeamercolor{frametitle}{fg=mytitle}
\setbeamercolor{title}{fg=mytitle}
\setbeamercolor{normal text}{fg=mytitle}
\setbeamercolor{structure}{fg=myaccent}

% ========================
% TÍTULO
% ========================
% Datos de portada
\title{Diseño Preliminar de un Satélite de Observación de la Tierra}
\author{Miguel Rosa Zazo \\ \footnotesize Tutor: Demetrio Zorita Gómez-Escolar }
\institute{Grado en Ingeniería Aeroespacial\\ \textbf{Universidad Politécnica de Madrid}}
\date{Julio 2025}

% ========================
% DOCUMENTO
% ========================
\begin{document}

% ===== SLIDE 1: TÍTULO =====
\begin{frame}[plain]
  \titlepage
  \begin{tikzpicture}[remember picture,overlay]
    % Logo derecho
    \node[anchor=south east,xshift=-0.5cm,yshift=0.5cm] at (current page.south east) {
      \includegraphics[width=3cm]{1959339-1955144650.png}
    };
    % Logo izquierdo (reemplaza este archivo por el logo de la escuela)
    \node[anchor=south west,xshift=0.5cm,yshift=0.5cm] at (current page.south west) {
      \includegraphics[width=3cm]{etsiae-1959139697.png}
    };
  \end{tikzpicture}
  \note[item]{Bienvenida a la audiencia.}
  \note[item]{Este proyecto es mi TFG para el grado en Ingeniería Aeroespacial.}
  \note[item]{Vamos a ver cómo diseñar un satélite de observación desde cero.}
\end{frame}

% ===== SLIDE 2: OBJETIVOS =====
\begin{frame}
  \frametitle{Introducción. Objetivos del Proyecto}
  \begin{itemize}
    \item Detección del CO$_{2}$ antropogénico: Bandas características.
    \item Requerimiento de MTF @ Nyquist $\geq$ 0,25 
    \item Requerimiento de SNR $\geq$ 400
    \item Cumplir una cobertura semanal sobre EE. UU (continental) con 1 dia cubierto por nubes, por cada 5 descubiertos.
    \item Minimizar la masa total puesta en orbita, con el fin de optimizar el coste de la misión
  \end{itemize}
  \note[item]{En esta sección explicamos el alcance del proyecto.}
  \note[item]{El objetivo no es fabricar el satélite, sino establecer un diseño preliminar.}
  
\end{frame}


% ===== SLIDE CAPÍTULO: Misiones semejantes =====
\begin{frame}[plain]
  \begin{center}
    \vspace{1.5cm}
    {\usebeamerfont{title}\color{myaccent}\Huge Misiones semejantes}
    
    \vspace{0.5cm}
    \rule{0.6\linewidth}{1pt}
    
    \vspace{1.5cm}
    {\Large\textit{Análisis comparativo de misiones existentes}}
  \end{center}


  \note[item]{Inicio del bloque dedicado a la carga útil del satélite.}
  \note[item]{Se introduce el diseño óptico, elección del detector y consideraciones espectrales.}
\end{frame}



% ===== SLIDE CAPÍTULO: CARGA DE PAGO =====
\begin{frame}[plain]
  \begin{center}
    \vspace{1.5cm}
    {\usebeamerfont{title}\color{myaccent}\Huge Carga de Pago}
    
    \vspace{0.5cm}
    \rule{0.6\linewidth}{1pt}
    
    \vspace{1.5cm}
    {\Large\textit{Diseño del sistema óptico y elección del detector}}
  \end{center}



  \note[item]{Inicio del bloque dedicado a la carga útil del satélite.}
  \note[item]{Se introduce el diseño óptico, elección del detector y consideraciones espectrales.}
\end{frame}


\begin{frame}{Diagrama}
%Llamada a funciones y entrada de parámetros: Master.m
\scriptsize
\hspace*{-1cm}
\begin{tikzpicture}[
    node distance=0.3cm and 1cm,
    box/.style={draw, rounded corners, minimum width=1cm, minimum height=1cm, align=center},
    ->, >=Stealth
]

% Nodos
\node[box] (mtf) at (-25,5) {MTF Calc.m};
\coordinate (puntoImagen) at ($(mtf.north west) + (-0.3cm,0.3cm)$);
\draw[->] (mtf.north west) -- (puntoImagen);
\node at ($(mtf.north west)+(-0.5cm,0.5cm)$) {
  \includegraphics[width=0.3cm]{png_filetype_icon_227876.png}%
  \hspace{0.1cm}
  \includegraphics[width=0.3cm]{txt_filetype_icon_227883.png}
};

\node[box, below=of mtf] (snr) {SNR Calc.m};
\coordinate (snrimg) at ($(snr.north west) + (-0.3cm,0.3cm)$);
\draw[->] (snr.north west) -- (snrimg);
\node at ($(snr.north west)+(-0.5cm,0.5cm)$) {
  \includegraphics[width=0.3cm]{png_filetype_icon_227876.png}%
  \hspace{0.1cm}
  \includegraphics[width=0.3cm]{txt_filetype_icon_227883.png}
};

\node[box, below=of snr] (coverage) {Coverage.m\\(\textit{wrapper})};
\coordinate (puntoImagen) at ($(coverage.north west) + (-0.3cm,0.3cm)$);
\draw[->] (coverage.north west) -- (puntoImagen);
\node at ($(coverage.north west)+(-0.5cm,0.5cm)$) {
  \includegraphics[width=0.3cm]{png_filetype_icon_227876.png}%
  \hspace{0.1cm}
  \includegraphics[width=0.3cm]{txt_filetype_icon_227883.png}
};

\node[box, below=of coverage] (revisit) {RevisitCalc.m};


\node[box, right=1cmof snr] (crossdata) {Cross Data.m};
\coordinate (puntoImagen) at ($(crossdata.north east) + (0.3cm,0.3cm)$);
\draw[->] (crossdata.north east) -- (puntoImagen);
\node at ($(crossdata.north east)+(0.5cm,0.5cm)$) {
  \includegraphics[width=0.3cm]{png_filetype_icon_227876.png}
  \hspace{0.1cm}
  \includegraphics[width=0.3cm]{txt_filetype_icon_227883.png}
};


\node[box, right=1cm of crossdata] (masa seca) {Masa Seca.m};
\coordinate (puntoImagen) at ($(masa seca.north east) + (0.3cm,0.3cm)$);
\draw[->] (masa seca.north east) -- (puntoImagen);
\node at ($(masa seca.north east)+(0.5cm,0.5cm)$) {
  \includegraphics[width=0.3cm]{png_filetype_icon_227876.png}
  \hspace{0.1cm}
  \includegraphics[width=0.3cm]{txt_filetype_icon_227883.png}
};


\node[box, right=1cm of masa seca] (masa total) {Masa Total.m};
\coordinate (puntoImagen) at ($(mtf.north west) + (-0.3cm,0.3cm)$);
\draw[->] (mtf.north west) -- (puntoImagen);
\node at ($(mtf.north west)+(-0.5cm,0.5cm)$) {
  \includegraphics[width=0.3cm]{png_filetype_icon_227876.png}%
  \hspace{0.1cm}
  \includegraphics[width=0.3cm]{txt_filetype_icon_227883.png}
};



\node[box, right=1cm of masa total] (config) {ConfigAnalysis.m};
\coordinate (puntoImagen) at ($(config.south) + (-0.0cm,-0.3cm)$);
\draw[->] (config.south) -- (puntoImagen);
\node at ($(config.south)+(-0.cm,-0.7cm)$) {
  \parbox{2.5cm}{%
    \centering
    \includegraphics[width=0.3cm]{csv_filetype_icon_227881-375492084.png} \hspace{0.1cm}
    \includegraphics[width=0.3cm]{txt_filetype_icon_227883.png} \\
    Lista de configuraciones
  }
};



% Flechas
\draw (mtf.east) -- (crossdata);
\node at ($ (mtf.east)!0.3!(crossdata) + (0,0.3) $) 
  {\includegraphics[width=0.3cm]{csv_filetype_icon_227881-375492084.png}};
  
\draw (snr) -- (crossdata);
\node at ($ (snr.east)!0.3!(crossdata) + (0,0.3) $) 
  {\includegraphics[width=0.3cm]{csv_filetype_icon_227881-375492084.png}};
  
\draw (coverage.east) -- (crossdata);
\node at ($ (coverage.east)!0.3!(crossdata) + (0,0.3) $) 
  {\includegraphics[width=0.3cm]{csv_filetype_icon_227881-375492084.png}};
\draw (crossdata) -- (masa seca);
\node at ($ (crossdata.east)!0.3!(masa seca) + (0,0.3) $) 
  {\includegraphics[width=0.3cm]{csv_filetype_icon_227881-375492084.png}};
\draw (masa seca) -- (masa total);
\node at ($ (masa seca.east)!0.3!(masa total) + (0,0.3) $) 
  {\includegraphics[width=0.3cm]{csv_filetype_icon_227881-375492084.png}};
  

\draw (masa total) -- (config);
\node at ($ (masa total.east)!0.3!(config) + (0,0.3) $) 
  {\includegraphics[width=0.3cm]{csv_filetype_icon_227881-375492084.png}};
  

\draw[->] ([xshift=2pt]coverage.south) -- ([xshift=2pt]revisit.north);
    \node[left=0.5cm of $(coverage.south)!0.5!(revisit.north)$] {max  days};
\draw[->] ([xshift=-2pt]revisit.north) -- ([xshift=-2pt]coverage.south);
\node at ($ ([xshift=-2pt]revisit.north)!0.5!([xshift=-1pt]coverage.south) + (1cm, 0) $) {Input data};
\end{tikzpicture}

\end{frame}

% ===== SLIDE CAPÍTULO: CARGA DE PAGO =====
\begin{frame}[plain]
  \begin{center}
    \vspace{1.5cm}
    {\usebeamerfont{title}\color{myaccent}\Huge Optimización de la masa}
    
    \vspace{0.5cm}
    \rule{0.6\linewidth}{1pt}
    
    \vspace{1.5cm}
    {\Large\textit{Selección de la altura órbital}}
  \end{center}



  \note[item]{Inicio del bloque dedicado a la carga útil del satélite.}
  \note[item]{Se introduce el diseño óptico, elección del detector y consideraciones espectrales.}
\end{frame}


% ===== SLIDE CAPÍTULO: CARGA DE PAGO =====
\begin{frame}[plain]
  \begin{center}
    \vspace{1.5cm}
    {\usebeamerfont{title}\color{myaccent}\Huge Subsistemas}
    
    \vspace{0.5cm}
    \rule{0.6\linewidth}{1pt}
    
    \vspace{1.5cm}
    {\Large\textit{XXXX}}
  \end{center}



  \note[item]{Inicio del bloque dedicado a la carga útil del satélite.}
  \note[item]{Se introduce el diseño óptico, elección del detector y consideraciones espectrales.}
\end{frame}

% ===== SLIDE CAPÍTULO: CARGA DE PAGO =====
\begin{frame}[plain]
  \begin{center}
    \vspace{1.5cm}
    {\usebeamerfont{title}\color{myaccent}\Huge Lanzadores}
    
    \vspace{0.5cm}
    \rule{0.6\linewidth}{1pt}
    
    \vspace{1.5cm}
    {\Large\textit{Lanzamiento y puesta en orbita de los satélites}}
  \end{center}



  \note[item]{Inicio del bloque dedicado a la carga útil del satélite.}
  \note[item]{Se introduce el diseño óptico, elección del detector y consideraciones espectrales.}
\end{frame}

% ===== SLIDE CAPÍTULO: CARGA DE PAGO =====
\begin{frame}[plain]
  \begin{center}
    \vspace{1.5cm}
    {\usebeamerfont{title}\color{myaccent}\Huge Lanzadores}
    
    \vspace{0.5cm}
    \rule{0.6\linewidth}{1pt}
    
    \vspace{1.5cm}
    {\Large\textit{Lanzamiento y puesta en orbita de los satélites}}
  \end{center}


  \note[item]{Inicio del bloque dedicado a la carga útil del satélite.}
  \note[item]{Se introduce el diseño óptico, elección del detector y consideraciones espectrales.}
\end{frame}

% ===== SLIDE CAPÍTULO: CARGA DE PAGO =====
\begin{frame}[plain]
  \begin{center}
    \vspace{1.5cm}
    {\usebeamerfont{title}\color{myaccent}\Huge Segmento Tierra}
    
    \vspace{0.5cm}
    \rule{0.6\linewidth}{1pt}
    
    \vspace{1.5cm}
    {\Large\textit{Generación y descarga de los datos}}
  \end{center}


  \note[item]{Inicio del bloque dedicado a la carga útil del satélite.}
  \note[item]{Se introduce el diseño óptico, elección del detector y consideraciones espectrales.}
\end{frame}

% ===== SLIDE FINAL: GRACIAS =====
\begin{frame}[plain]
  \begin{center}
    \vspace{2cm}
    {\usebeamerfont{title}\color{myaccent}\Huge ¡Gracias!}
    
    \vspace{1cm}
    {\Large\textit{¿Preguntas?}}

    \vfill
    \texttt{\textcolor{mycode}{tunombre.github.io/tfg-observacion}}
  \end{center}
  
  \begin{tikzpicture}[remember picture,overlay]
    % Logo derecho
    \node[anchor=south east,xshift=-0.5cm,yshift=0.5cm] at (current page.south east) {
      \includegraphics[width=3cm]{1959339-1955144650.png}
    };
    % Logo izquierdo
    \node[anchor=south west,xshift=0.5cm,yshift=0.5cm] at (current page.south west) {
      \includegraphics[width=3cm]{etsiae-1959139697.png}
    };
  \end{tikzpicture}

  \note[item]{Agradecer la atención del tribunal y la audiencia.}
  \note[item]{Recordar que el repositorio contiene simulaciones y memoria completa.}
  \note[item]{Momento para responder preguntas.}
\end{frame}

\begin{frame}[allowframebreaks]
  \frametitle{Referencias}
  \printbibliography
\end{frame}

\end{document}